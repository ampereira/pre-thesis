%!TEX root = ../main.tex
\chapter{Research Plan}
\label{thesis_planning}

\itodo{Datas ainda por definir e a ordem ainda não está definida}
\itodo{As tarefas a curto prazo estão definidas enquanto que as restantes ainda são bastante abstractas}
\itodo{Faltam tarefas para além do protótipo (além das relacionadas com aceleradores, que serão deixadas para depois de Março de 2015)}

The research plan for the PhD thesis work is as follows:

\begin{itemize}
	\item Requirements elicitation of all physics features to implement on the framework together with the LIP researchers.
	\item Validate the proposed framework design and new data analysis programming model.
	\item Redesign of the features already implemented in LipMiniAnalysis to fit the new framework requirements.
	\item Assess and compare the performance of various C++ collections to store various events on memory.
	\item Extend the current event data structure to fit the requirements of all data analysis:
	\begin{itemize}
		\item By adding the all variables in the ROOT input data files, or;
		\item By the user defining which variables are needed for a given analysis, and the data structure being automatically created using that information;
	\end{itemize}
	\item Extend the I/O features available in LipMiniAnalysis:
	\begin{itemize}
		\item By reading multiple input data files in parallel;
		\item By build the event data structure in parallel;
	\end{itemize}
	\item Assess and compare different performance models for irregular workload balance in:
	\begin{itemize}
		\item Homogeneous systems;
		\item Heterogeneous systems initially using the \intel Xeon Phi;
	\end{itemize}
	\item Support for hybrid process/thread automatic parallelization.
	\item Present the first framework prototype in March 2015.
\end{itemize}
