%!TEX root = ../main.tex
\chapter*{Abstract}
\label{abstract}

Most event data analysis tasks in the ATLAS Experiment require intensive data accessing and processing, where the computing performance of the applications constrain both the analysis throughput and the accuracy of the studied physics properties. This work focus on the compute bound issues at the last stage of ATLAS detector data analysis.

The goal for this work is to provide an efficient unified particle physics framework to aid the development of data analysis applications, within the LIP research group. It will replace the LipMiniAnalysis skeleton library, developed by LIP, and offer a more robust tool that efficiently uses the available computational resources of multi-CPU platforms coupled with hardware accelerators, while abstracting the intrinsic complexities of theses systems to the user. It redesigns the current programming model of data analysis, by providing efficient data structures, parallelization mechanisms, and managing all data analysis common functionalities, freeing the user to only code the specifics of each analysis.

Preliminary work identified the LipMiniAnalysis skeleton library inefficiencies, and a design of the new particle physics framework was proposed. It overcomes the data structure inefficiencies of LipMiniAnalysis and allows for a kernel-like programming model of data analysis, while automatically handling the data I/O, event data structure creation, and efficient load balancing among multiple CPU cores and hardware accelerators, such as GPUs and the \intel Xeon Phi. Most computing features of the framework were already detailed in the design specification.

Several prototypes of some of the framework core features were implemented and tested in the current LipMiniAnalysis skeleton library. This document presents a new data structure capable of holding multiple events in memory, an automatic parallelization of the even processing in a multi-CPU computing node, and a dynamic interface generator that provides a \textit{plug-and-play} integration of the framework with legacy data analysis code.
